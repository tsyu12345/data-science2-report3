\documentclass[dvipdfmx]{jsarticle}
\usepackage[T1]{fontenc}
\usepackage[dvipdfmx]{hyperref}
\usepackage{lmodern}
\usepackage{latexsym}
\usepackage{amsfonts}
\usepackage{amssymb}
\usepackage{mathtools}
\usepackage{amsthm}
\usepackage{multirow}
\usepackage{graphicx}
\usepackage{wrapfig}
\usepackage{here}
\usepackage{float}
\usepackage{ascmac}
\usepackage{url}

\title{Groceriesに関するR言語を使用した頻出パターン抽出及び相関ルール分析}
\author{文理学部情報科学科\\5419045 高林 秀}
\date{\today}

\begin{document}

\maketitle

\begin{abstract}
  本稿では、今年度データ科学2で学習した「頻出パターン抽出」及び「相関ルール分析」の手法を使用して、R言語のライブラリであるarulesに付属しているデータGroceriesを対象とした頻出パターン抽出、相関ルール分析を行うものである。

\end{abstract}

\section{目的}
本稿では実際に、R言語を使用しライブラリarules付属のデータであるGroceriesの頻出パターン抽出、相関ルール分析を行うことで、本年度データ科学2で学習した頻出パターン、相関ルール分析の手法への理解を深め、その定着を図ることを目的とする。また、1年次に学習したlatexを用いたPDF作成の復習も兼ねるものである。
\section{理論説明}
今回の実験で用いた、計算理論をそれぞれ説明する。
\subsection{バスケット分析}
\subsection{頻出パターン・相関ルール分析の概要}
\subsection{頻出パターンの計算法}
\subsection{支持度}
\subsection{確信度}
\subsection{パターン空間について}
\subsection{(補足)深さ優先探索・幅優先探索}
\subsection{バックトラック法}
\subsection{アプリオリアルゴリズム}
\subsection{相関ルール抽出の計算法}
\subsection{頻出パターン抽出の問題点}
\subsection{相関ルール分析の評価基準}

\section{計算機実験}
\subsection{実験準備}
  \subsubsection{実験環境}
  今回の実験は仮想マシン上でR言語を起動し行った。下記に実験時の環境を示す。
  \begin{itemize}
    \item ホストOS:Window10 Home Ver.20H2
    \item 仮想OS:Ubuntu 20.04.2 LTS
    \item CPU:Intel(R)Core(TM)i7-9700K @ 3.6GHz
    \item GPU:Nvidia Geforce RTX2070 OC @ 8GB
    \item ホストRAM:16GB
    \item 仮想RAM:4GB
  \end{itemize}
\subsubsection{実験データ}
\subsubsection{R言語での頻出パターン・相関ルール分析の手法}
\subsection{実験結果}
\subsection{結果の説明}

\section{考察}
\section{まとめ}

\begin{thebibliography}{99}
\end{thebibliography}
\end{document}
